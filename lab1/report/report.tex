\documentclass{article}
\usepackage[top=2cm,bottom=2cm,left=3cm,right=3cm]{geometry}
\usepackage{enumitem}
\usepackage{xcolor}
\usepackage{minted}
\usepackage[most]{tcolorbox}
\tcbuselibrary{minted,breakable}
\usepackage{fontspec}
\usepackage{polyglossia}
\setdefaultlanguage{ukrainian}
\setotherlanguages{english}
\usepackage{Alegreya}
\usepackage{FiraMono}
\newfontfamily\cyrillicfont{Alegreya}
\newfontfamily\cyrillicfontsf{Alegreya Sans}
\newfontfamily\cyrillicfonttt{DejaVu Sans Mono}
\usepackage{sciffi-python}
\usepackage{sciffi-memo}

\usemintedstyle{emacs}

\newtcolorbox{CodeBox}[1][]{
    breakable,
    enhanced,
    colback=gray!10,
    colframe=gray!70,
    fonttitle=\bfseries,
    title=#1,
    sharp corners,
    boxrule=0.8pt,
    top=2mm,
    bottom=2mm,
    left=2mm,
    right=2mm,
    fontupper=\footnotesize,
}

\newtcolorbox{ResultBox}[1][]{
    breakable,
    enhanced,
    colback=gray!5,
    colframe=gray!50,
    fonttitle=\bfseries,
    title=#1,
    sharp corners,
    boxrule=0.6pt,
    top=1mm,
    bottom=1mm,
    left=2mm,
    right=2mm,
    fontupper=\footnotesize,
}

\begin{document}
    \begin{titlepage}
        \begin{center}
            {\Large
                МІНІСТЕРСТВО ОСВІТИ ТА НАУКИ УКРАЇНИ

                НАЦІОНАЛЬНИЙ ТЕХНІЧНИЙ УНІВЕРСИТЕТ УКРАЇНИ
                «КИЇВСЬКИЙ ПОЛІТЕХНІЧНИЙ ІНСТИТУТ ІМЕНІ ІГОРЯ СІКОРСЬКОГО»

                ФАКУЛЬТЕТ ІНФОРМАТИКИ ТА ОБЧИСЛЮВАЛЬНОЇ ТЕХНІКИ

                КАФЕДРА ІНФОРМАЦІЙНИХ СИСТЕМ ТА ТЕХНОЛОГІЙ
            }


            \vspace{60mm}
            {\large
                \textbf{ЗВІТ}

                \vspace{5mm}

                \textbf{До лабораторної роботи 1}

                \vspace{5mm}

                з дисципліни «Технології Розробки Програмного Забезпечення»
            }

            На тему «Музичний Магазин»

            варіант \textnumero 14 (\textnumero 4)
        \end{center}

        \vfill
        \hfill
        \begin{minipage}[t]{0.35\textwidth}
            \textbf{Виконав:}
        \end{minipage}

        \vfill

        \begin{center}
            {\bf
                Київ

                КПІ Ім. Ігоря Сікорського

                2025
            }
        \end{center}
    \end{titlepage}
    \newpage

    Тема: створення бази даних, таблиць бази даних та заповнення таблиць даними.

    Мета: отримати навички використання команди створення, зміни, видалення,
    процедур перейменування виведення відомостей про бази даних; таблиці бази
    даних та заповнення таблиць БД даними.

    Навички що будуть здобуті: вміння створювати, змінювати і заповнювати
    даними базу даних та таблиці у ній.

    \section*{Хід виконання:}
    \begin{sciffi}{python}
        from pathlib import Path
        from subprocess import run, PIPE
        from collections.abc import Sequence
        from contextlib import contextmanager
        
        @contextmanager
        def env(name: str, *args: str):
            print(f"\\begin{{{name}}}" + "".join(args))
            try:
                yield
            finally:
                print(f"\\end{{{name}}}")

        def sh(*args: str | Path) -> str:
            return run(
                tuple(map(str, args)),
                stdout=PIPE,
                stderr=PIPE,
                text=True,
            ).stdout

        scripts = Path("./scripts/")
        flow: Sequence[tuple[Path, Sequence[str]]] = (
            (scripts / "db.sql", ("--autocommit", "--no-echo")),
            (scripts / "renamedb.sql", ("--no-echo",)),
            (scripts / "spaceused.sql", ("--no-echo",)),
            (scripts / "schema.sql", ("--no-echo",)),
            (scripts / "discount.sql", ("--no-echo",)),
            (scripts / "seed.sql", ("--no-echo",)),
            (scripts / "select.sql", ("--no-echo",)),
        )

        for script, args in flow:
            source = script.read_text(encoding="utf-8")
            output = sh("tsqlexec", *args, script)

            print(f"Script path: {script}")
            with env("CodeBox", "[SQL Source]"):
                with env("minted", "[breaklines,linenos]", "{sql}"):
                    print(source, end="")
            print()

            if not output.strip():
                continue

            print("Result:")
            with env("ResultBox", "[Output]"):
                with env("minted", "[breaklines]", "{text}"):
                    print(output, end="")
    \end{sciffi}

    \newpage
    \section*{Контрольні питання}
    \begin{enumerate}[label=\arabic*., leftmargin=*]
        \item \textbf{Пріоритет виконання операторів.}

        Пріоритет операторів в SQL визначає порядок, у якому обчислюються
        вирази. Оператори з вищим пріоритетом виконуються першими. Якщо
        оператори мають однаковий пріоритет, вони виконуються зліва направо.
        Пріоритет від найвищого до найнижчого:
        \begin{enumerate}
            \item \texttt{()} — дужки для зміни пріоритету.
            \item \texttt{*}, \texttt{/}, \texttt{\%} — арифметичні оператори множення, ділення, остача від ділення.
            \item \texttt{+}, \texttt{-} — арифметичні оператори додавання, віднімання.
            \item Оператори порівняння: \texttt{=}, \texttt{>}, \texttt{<}, \texttt{>=}, \texttt{<=}, \texttt{<>}, \texttt{!=}.
            \item \texttt{NOT} — логічне заперечення.
            \item \texttt{AND} — логічне "І".
            \item \texttt{OR} — логічне "АБО".
        \end{enumerate}

        \item \textbf{Правила присвоювання імен об’єктам бази даних.}

        Імена об'єктів (баз даних, таблиць, стовпців) називаються
        ідентифікаторами. Існують такі правила:
        \begin{itemize}
            \item \textbf{Довжина:} Зазвичай до 128 символів (у SQL Server).
            \item \textbf{Перший символ:} Має бути літера, символ підкреслення \texttt{\_}, \texttt{@} (для локальних змінних) або \texttt{\#} (для тимчасових таблиць).
            \item \textbf{Наступні символи:} Можуть бути літери, цифри, символи \texttt{@}, \texttt{\$}, \texttt{\#}, \texttt{\_}.
            \item \textbf{Пробіли та спецсимволи:} Якщо ім'я містить пробіли або зарезервовані слова, його слід брати у подвійні лапки \texttt{" "} або квадратні дужки \texttt{[]}.
            \item \textbf{Унікальність:} Імена таблиць мають бути унікальними в межах схеми, а імена стовпців — унікальними в межах таблиці.
        \end{itemize}
        
        \item \textbf{Синтаксис команди створення бази даних.}

        Команда \texttt{CREATE DATABASE} використовується для створення нової бази даних.
        \begin{CodeBox}[Базовий синтаксис]
        \begin{minted}[breaklines]{sql}
CREATE DATABASE database_name;
        \end{minted}
        \end{CodeBox}
        
        \item \textbf{Як можна отримати відомості про базу даних?}

        Для цього існують системні інструменти:
        \begin{itemize}
            \item Системна процедура \texttt{sp\_helpdb} (T-SQL):
            \begin{CodeBox}[SQL]
            \begin{minted}[breaklines]{sql}
EXEC sp_helpdb 'database_name';
            \end{minted}
            \end{CodeBox}
            \item Системні представлення (Catalog Views):
            \begin{CodeBox}[SQL]
            \begin{minted}[breaklines]{sql}
SELECT * FROM sys.databases WHERE name = 'database_name';
            \end{minted}
            \end{CodeBox}
        \end{itemize}

        \item \textbf{Як застосовується команда ALTER DATABASE?}

        Команда \texttt{ALTER DATABASE} використовується для зміни властивостей існуючої бази даних, наприклад, для перейменування, додавання файлів або зміни параметрів.
        \begin{CodeBox}[Приклад зміни імені]
        \begin{minted}[breaklines]{sql}
ALTER DATABASE OldDatabaseName MODIFY NAME = NewDatabaseName;
        \end{minted}
        \end{CodeBox}
        \begin{CodeBox}[Приклад додавання файлу]
        \begin{minted}[breaklines]{sql}
ALTER DATABASE MyDatabase ADD FILE (NAME = 'NewDataFile', FILENAME = 'C:\path\to\file.ndf');
        \end{minted}
        \end{CodeBox}

        \item \textbf{Як можна перейменувати базу даних?}

        Перейменувати базу даних можна за допомогою команди \texttt{ALTER DATABASE} або застарілої процедури \texttt{sp\_renamedb}. Сучасний спосіб є кращим.
        \begin{CodeBox}[Перейменування бази даних]
        \begin{minted}[breaklines]{sql}
ALTER DATABASE OldName MODIFY NAME = NewName;
        \end{minted}
        \end{CodeBox}
        
        \item \textbf{Синтаксис команди видалення бази даних.}

        Команда \texttt{DROP DATABASE} видаляє базу даних та всі її об’єкти назавжди.
        \begin{CodeBox}[Синтаксис]
        \begin{minted}[breaklines]{sql}
DROP DATABASE database_name;
        \end{minted}
        \end{CodeBox}
        Ця операція є незворотною, тому її слід використовувати з обережністю.
        
        \item \textbf{Синтаксис команди створення таблиці бази даних.}

        Команда \texttt{CREATE TABLE} створює нову таблицю у поточній базі даних.
        \begin{CodeBox}[Синтаксис]
        \begin{minted}[breaklines]{sql}
CREATE TABLE table_name (
    column1_name data_type [constraints],
    column2_name data_type [constraints],
    ...
    [table_constraints]
);
        \end{minted}
        \end{CodeBox}
        
        \item \textbf{Поняття та синтаксис Primary key та Foreign key.}

        \begin{itemize}
            \item \textbf{Primary Key (первинний ключ):} Це стовпець або набір стовпців, який унікально ідентифікує кожен запис у таблиці. Значення в ньому не можуть бути \texttt{NULL} і не можуть повторюватися.
            \begin{CodeBox}[Синтаксис Primary Key]
            \begin{minted}[breaklines]{sql}
CREATE TABLE Students (
    StudentID INT PRIMARY KEY,
    FirstName VARCHAR(50) NOT NULL
);
            \end{minted}
            \end{CodeBox}
            
            \item \textbf{Foreign Key (зовнішній ключ):} Це стовпець у одній таблиці, який посилається на первинний ключ в іншій таблиці. Він створює зв'язок між таблицями і забезпечує посилальну цілісність.
            \begin{CodeBox}[Синтаксис Foreign Key]
            \begin{minted}[breaklines]{sql}
CREATE TABLE Orders (
    OrderID INT PRIMARY KEY,
    StudentID INT,
    FOREIGN KEY (StudentID) REFERENCES Students(StudentID)
);
            \end{minted}
            \end{CodeBox}
        \end{itemize}

        \item \textbf{Синтаксис та правила застосування IDENTITY.}

        \texttt{IDENTITY} — це властивість стовпця (в T-SQL), яка автоматично генерує послідовні числові значення при додаванні нових рядків.
        \begin{CodeBox}[Синтаксис]
        \begin{minted}[breaklines]{sql}
IDENTITY(seed, increment)
-- seed - початкове значення, increment - крок
        \end{minted}
        \end{CodeBox}
        Правила: може бути лише один \texttt{IDENTITY}-стовпець на таблицю; тип даних має бути числовим; при вставці даних значення для цього поля вказувати не потрібно.

        \item \textbf{Як можна отримати відомості про таблицю бази даних?}

        Використовується системна процедура \texttt{sp\_help}.
        \begin{CodeBox}[SQL]
        \begin{minted}[breaklines]{sql}
EXEC sp_help 'table_name';
        \end{minted}
        \end{CodeBox}
        Вона показує структуру таблиці, стовпці, їх типи даних, індекси та обмеження.

        \item \textbf{Як застосовується команда ALTER TABLE?}

        Команда \texttt{ALTER TABLE} дозволяє змінювати структуру існуючої таблиці.
        \begin{itemize}
            \item \textbf{Додати стовпець:} \texttt{ALTER TABLE table\_name ADD column\_name data\_type;}
            \item \textbf{Видалити стовпець:} \texttt{ALTER TABLE table\_name DROP COLUMN column\_name;}
            \item \textbf{Змінити тип даних:} \texttt{ALTER TABLE table\_name ALTER COLUMN column\_name new\_data\_type;}
        \end{itemize}
        
        \item \textbf{Як можна перейменувати таблицю бази даних?}

        У T-SQL для цього використовується системна процедура \texttt{sp\_rename}.
        \begin{CodeBox}[Синтаксис]
        \begin{minted}[breaklines]{sql}
EXEC sp_rename 'OldTableName', 'NewTableName';
        \end{minted}
        \end{CodeBox}
        
        \item \textbf{Синтаксис команди видалення таблиці бази даних.}

        Команда \texttt{DROP TABLE} повністю видаляє таблицю, її структуру та дані.
        \begin{CodeBox}[Синтаксис]
        \begin{minted}[breaklines]{sql}
DROP TABLE table_name;
        \end{minted}
        \end{CodeBox}
        
        \item \textbf{Синтаксис команди заповнення таблиці даними.}

        Для додавання нових рядків у таблицю використовується команда \texttt{INSERT INTO}.
        \begin{CodeBox}[Синтаксис]
        \begin{minted}[breaklines]{sql}
-- Вставка значень для всіх стовпців
INSERT INTO table_name VALUES (value1, value2, ...);

-- Вставка значень для конкретних стовпців
INSERT INTO table_name (column1, column2) VALUES (value1, value2);
        \end{minted}
        \end{CodeBox}
        
        \item \textbf{Порядок заповнення таблиць даними.}

        Якщо таблиці пов'язані зовнішніми ключами, їх потрібно заповнювати у правильному порядку:
        \begin{enumerate}
            \item Спочатку заповнюються "головні" (батьківські) таблиці, на які посилаються зовнішні ключі.
            \item Потім заповнюються "дочірні" (залежні) таблиці.
        \end{enumerate}
        Це необхідно для дотримання посилальної цілісності.
        
        \item \textbf{Правила заповнення основними типами даних (Int, Char, Float, Date).}

        \begin{itemize}
            \item \textbf{Int, Float (числові):} Записуються без лапок. Наприклад: \texttt{123}, \texttt{199.99}.
            \item \textbf{Char, VarChar (символьні):} Значення беруться в одинарні лапки. Наприклад: \texttt{'Київ'}.
            \item \textbf{Date, DateTime (дата/час):} Значення беруться в одинарні лапки у форматі \texttt{'YYYY-MM-DD'}. Наприклад: \texttt{'2023-10-27'}.
        \end{itemize}
        
        \item \textbf{Різниця між заповненням всіх полів та вибіркової кількості полів таблиці.}

        \begin{itemize}
            \item \textbf{Заповнення всіх полів (\texttt{INSERT ... VALUES ...}):}
                Вимагає надання значень для кожного стовпця в тому
                порядку, в якому вони визначені в таблиці. Цей підхід є крихким
                до змін структури таблиці.

            \item \textbf{Заповнення вибіркових полів (\texttt{INSERT ... (col1) VALUES ...}):}
                Ви чітко вказуєте, в які стовпці
                вставляєте дані. Це надійніший та більш читабельний підхід,
                оскільки він не залежить від порядку стовпців.
        \end{itemize}
        
        \item \textbf{Правила заповнення вибіркової кількості полів таблиці.}

        \begin{enumerate}
            \item Обов'язково вказувати список стовпців у дужках після імені таблиці.
            \item Список значень у \texttt{VALUES} повинен відповідати списку стовпців за кількістю та порядком.
            \item Стовпці, які не були вказані, повинні дозволяти \texttt{NULL} або мати значення за замовчуванням (\texttt{DEFAULT}).
        \end{enumerate}
        
        \item \textbf{Правила заповнення даними таблиць, які мають автоінкрементне поле.}

        Основне правило: при звичайній вставці даних (\texttt{INSERT}) ви
        \textbf{не вказуєте} автоінкрементний стовпець у списку стовпців і
        \textbf{не передаєте} для нього значення. СКБД згенерує його
        автоматично.

        \begin{CodeBox}[Приклад]
        \begin{minted}[breaklines]{sql}
-- Таблиця: Products (ProductID INT IDENTITY, ProductName VARCHAR)
INSERT INTO Products (ProductName) VALUES ('Ноутбук'); 
-- ProductID буде згенеровано автоматично
        \end{minted}
        \end{CodeBox}
    \end{enumerate}
\end{document}
